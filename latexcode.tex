\documentclass[journal,12pt,twocolumn]{IEEEtran}

\usepackage{setspace}
\usepackage{gensymb}

\singlespacing


\usepackage[cmex10]{amsmath}

\usepackage{amsthm}

\usepackage{mathrsfs}
\usepackage{txfonts}
\usepackage{stfloats}
\usepackage{bm}
\usepackage{cite}
\usepackage{cases}
\usepackage{subfig}

\usepackage{longtable}
\usepackage{multirow}

\usepackage{enumitem}
\usepackage{mathtools}
\usepackage{steinmetz}
\usepackage{tikz}
\usepackage{circuitikz}
\usepackage{verbatim}
\usepackage{tfrupee}
\usepackage[breaklinks=true]{hyperref}
\usepackage{graphicx}
\usepackage{tkz-euclide}

\usetikzlibrary{calc,math}
\usepackage{listings}
    \usepackage{color}                                            %%
    \usepackage{array}                                            %%
    \usepackage{longtable}                                        %%
    \usepackage{calc}                                             %%
    \usepackage{multirow}                                         %%
    \usepackage{hhline}                                           %%
    \usepackage{ifthen}                                           %%
    \usepackage{lscape}     
\usepackage{multicol}
\usepackage{chngcntr}

\DeclareMathOperator*{\Res}{Res}

\renewcommand\thesection{\arabic{section}}
\renewcommand\thesubsection{\thesection.\arabic{subsection}}
\renewcommand\thesubsubsection{\thesubsection.\arabic{subsubsection}}

\renewcommand\thesectiondis{\arabic{section}}
\renewcommand\thesubsectiondis{\thesectiondis.\arabic{subsection}}
\renewcommand\thesubsubsectiondis{\thesubsectiondis.\arabic{subsubsection}}


\hyphenation{op-tical net-works semi-conduc-tor}
\def\inputGnumericTable{}                                 %%

\lstset{
%language=C,
frame=single, 
breaklines=true,
columns=fullflexible
}
\begin{document}


\newtheorem{theorem}{Theorem}[section]
\newtheorem{problem}{Problem}
\newtheorem{proposition}{Proposition}[section]
\newtheorem{lemma}{Lemma}[section]
\newtheorem{corollary}[theorem]{Corollary}
\newtheorem{example}{Example}[section]
\newtheorem{definition}[problem]{Definition}

\newcommand{\BEQA}{\begin{eqnarray}}
\newcommand{\EEQA}{\end{eqnarray}}
\newcommand{\define}{\stackrel{\triangle}{=}}
\bibliographystyle{IEEEtran}

\providecommand{\mbf}{\mathbf}
\providecommand{\pr}[1]{\ensuremath{\Pr\left(#1\right)}}
\providecommand{\qfunc}[1]{\ensuremath{Q\left(#1\right)}}
\providecommand{\sbrak}[1]{\ensuremath{{}\left[#1\right]}}
\providecommand{\lsbrak}[1]{\ensuremath{{}\left[#1\right.}}
\providecommand{\rsbrak}[1]{\ensuremath{{}\left.#1\right]}}
\providecommand{\brak}[1]{\ensuremath{\left(#1\right)}}
\providecommand{\lbrak}[1]{\ensuremath{\left(#1\right.}}
\providecommand{\rbrak}[1]{\ensuremath{\left.#1\right)}}
\providecommand{\cbrak}[1]{\ensuremath{\left\{#1\right\}}}
\providecommand{\lcbrak}[1]{\ensuremath{\left\{#1\right.}}
\providecommand{\rcbrak}[1]{\ensuremath{\left.#1\right\}}}
\theoremstyle{remark}
\newtheorem{rem}{Remark}
\newcommand{\sgn}{\mathop{\mathrm{sgn}}}
\providecommand{\abs}[1]{\left\vert#1\right\vert}
\providecommand{\res}[1]{\Res\displaylimits_{#1}} 
\providecommand{\norm}[1]{\left\lVert#1\right\rVert}
%\providecommand{\norm}[1]{\lVert#1\rVert}
\providecommand{\mtx}[1]{\mathbf{#1}}
\providecommand{\mean}[1]{E\left[ #1 \right]}
\providecommand{\fourier}{\overset{\mathcal{F}}{ \rightleftharpoons}}
%\providecommand{\hilbert}{\overset{\mathcal{H}}{ \rightleftharpoons}}
\providecommand{\system}{\overset{\mathcal{H}}{ \longleftrightarrow}}
	%\newcommand{\solution}[2]{\textbf{Solution:}{#1}}
\newcommand{\solution}{\noindent \textbf{Solution: }}
\newcommand{\cosec}{\,\text{cosec}\,}
\providecommand{\dec}[2]{\ensuremath{\overset{#1}{\underset{#2}{\gtrless}}}}
\newcommand{\myvec}[1]{\ensuremath{\begin{pmatrix}#1\end{pmatrix}}}
\newcommand{\mydet}[1]{\ensuremath{\begin{vmatrix}#1\end{vmatrix}}}

\numberwithin{equation}{subsection}

\makeatletter
\@addtoreset{figure}{problem}
\makeatother
\let\StandardTheFigure\thefigure
\let\vec\mathbf

\renewcommand{\thefigure}{\theproblem}

\def\putbox#1#2#3{\makebox[0in][l]{\makebox[#1][l]{}\raisebox{\baselineskip}[0in][0in]{\raisebox{#2}[0in][0in]{#3}}}}
     \def\rightbox#1{\makebox[0in][r]{#1}}
     \def\centbox#1{\makebox[0in]{#1}}
     \def\topbox#1{\raisebox{-\baselineskip}[0in][0in]{#1}}
     \def\midbox#1{\raisebox{-0.5\baselineskip}[0in][0in]{#1}}
\vspace{3cm}
\title{Assignment 8}
\author{KUSUMA PRIYA\\EE20MTECH11007}

\maketitle
\newpage

\bigskip
\renewcommand{\thefigure}{\theenumi}
\renewcommand{\thetable}{\theenumi}
Download all python codes from 
\begin{lstlisting}
https://github.com/KUSUMAPRIYAPULAVARTY/assignment8/tree/master/codes
\end{lstlisting}
%
and latex-tikz codes from 
%
\begin{lstlisting}
https://github.com/KUSUMAPRIYAPULAVARTY/assignment8
\end{lstlisting}
%
 
 \section{QUESTION}
Check if the lines $L_1,L_2$ are skew.If so, find the closest points on those lines using Singular Value Decomposition(SVD)
\begin{align}
L_1: \vec{x}=\myvec{2\\-5\\1}+\lambda_1\myvec{3\\2\\6}\\
L_2: \vec{x}=\myvec{7\\-6\\0}+\lambda_1\myvec{1\\2\\2}
\end{align}

%
\section {Explanation}
The matrix \vec{M} of dimensions $(m \times n)$ can be decomposed using SVD as 
\begin{align}
\vec{M}=\vec{U}\vec{S}\vec{V}^T\label{1}
\end{align}
where,columns of $\vec{U}_{(m \times m)}$ are eigen vectors of $\vec{M}\vec{M}^T$\\
columns of $\vec{V}_{(n \times n)}$ are eigen vectors of $\vec{M}^T\vec{M}$\\
$\vec{S}$ is a diagonal matrix containing singular values of $\vec{M}$. Also, $\vec{U}$ and $\vec{V}$ are orthogonal matrices
\begin{align}
    \vec{U}\vec{U}^T=\vec{U}^T\vec{U}=\vec{I}\\
     \vec{V}\vec{V}^T=\vec{V}^T\vec{V}=\vec{I}
\end{align}
\section{Solution}
Given line equations intersect if
\begin{align}
 \myvec{2\\-5\\1}+\lambda_1\myvec{3\\2\\6}=\myvec{7\\-6\\0}+\lambda_2\myvec{1\\2\\2}
\end{align}
This can be written as 
\begin{align}
    \myvec{3&1\\2&2\\6&2}\vec{x}=\myvec{5\\-1\\-1}\\
    \vec{M}\vec{x}=\vec{b}
\end{align}
The augmented matrix is 
\begin{align}
    \myvec{3&1&5\\2&2&-1\\6&2&-1}\xleftrightarrow[R_2\leftarrow R_2-R_1\times \frac{2}{3}]{R_3\leftarrow R_3-2\times R_1}\myvec{3&1&5\\0&\frac{5}{3}&-\frac{13}{3}\\0&0&-11}
\end{align}
So, the given pair of lines do not intersect and also their direction vectors are not parallel. Hence they are skew lines.\\
To find \vec{U},
\begin{align}
    \vec{M}\vec{M}^T= \myvec{3&1\\2&2\\6&2}\myvec{3&2&6\\1&2&2}=\myvec{10&8&20\\8&8&16\\20&16&40}
    \end{align}
    To calculate its Eigen values,
    \begin{align}
    \mydet{10-\lambda&8&20\\8&8-\lambda&16\\20&16&40-\lambda}=0\\
    \implies \lambda^3+58\lambda^2+80\lambda=0\\
    \lambda_1=29-\sqrt{761},\lambda_2=29+\sqrt{761},\lambda_3=0
\end{align}
with corresponding Eigen vectors as 
\begin{align}
\vec{u_1}=\frac{1}{\sqrt{\left(\frac{1}{2}\right)^2+1+\left(\frac{21+\sqrt{761}}{16}\right)^2}}\myvec{\frac{1}{2}\\\frac{-21-\sqrt{761}}{16}\\1}\\
\vec{u_2}=\frac{1}{\sqrt{\left(\frac{1}{2}\right)^2+1+\left(\frac{-21+\sqrt{761}}{16}\right)^2}}\myvec{\frac{1}{2}\\\frac{-21+\sqrt{761}}{16}\\1}\\
\vec{u_3}=\frac{1}{\sqrt{(-2)^2+1}}\myvec{-2\\0\\1}
\end{align}
Solving, the $\vec{U}$ matrix becomes
\begin{align}
    \vec{U}=\myvec{\frac{8}{\sqrt{1522+42\sqrt{761}}}&\frac{8}{\sqrt{1522-42\sqrt{761}}}&-\frac{2}{\sqrt{5}}\\\frac{-21-\sqrt{761}}{\sqrt{1522+42\sqrt{761}}}&\frac{-21+\sqrt{761}}{\sqrt{1522-42\sqrt{761}}}&0\\\frac{16}{\sqrt{1522+42\sqrt{761}}}&\frac{16}{\sqrt{1522-42\sqrt{761}}}&\frac{1}{\sqrt{5}}}
\end{align}
Also, from the obtained Eigen values,the $\vec{S}$ matrix becomes
\begin{align}
    \vec{S}=\myvec{\sqrt{29-\sqrt{761}}&0\\0&\sqrt{29+\sqrt{761}}\\0&0}
\end{align}
The Moore-Penrose pseudo inverse of $\vec{S}$ is given by
\begin{align}
    \vec{S}_+=\myvec{\frac{1}{\sqrt{29-\sqrt{761}}}&0&0\\0&\frac{1}{\sqrt{29+\sqrt{761}}}&0}\label{2}
\end{align}
Now to find $\vec{V}$,\\
Rewriting \eqref{1}
\begin{align}
    \vec{V}=(\vec{M}^T\vec{U})\vec{S_+}^T\label{3}
    \end{align}
    $ \vec{M}^T\vec{U}$ becomes
    \begin{align}
   \myvec{3&2&6\\1&2&2}\\\myvec{\frac{8}{\sqrt{1522+42\sqrt{761}}}&\frac{8}{\sqrt{1522-42\sqrt{761}}}&-\frac{2}{\sqrt{5}}\\\frac{-21-\sqrt{761}}{\sqrt{1522+42\sqrt{761}}}&\frac{-21+\sqrt{761}}{\sqrt{1522-42\sqrt{761}}}&0\\\frac{16}{\sqrt{1522+42\sqrt{761}}}&\frac{16}{\sqrt{1522-42\sqrt{761}}}&\frac{1}{\sqrt{5}}}\\
   =\myvec{\frac{78-2\sqrt{761}}{\sqrt{1522+42\sqrt{761}}}&\frac{78+2\sqrt{761}}{\sqrt{1522-42\sqrt{761}}}&0\\\frac{-2-2\sqrt{761}}{\sqrt{1522+42\sqrt{761}}}&\frac{-2+2\sqrt{761}}{\sqrt{1522-42\sqrt{761}}}&0}\label{4}
\end{align}
Therefore from \eqref{2},\eqref{3},\eqref{4},
\begin{align}
 \vec{V}=\myvec{\frac{78-2\sqrt{761}}{\sqrt{1522+42\sqrt{761}}\sqrt{29-\sqrt{761}}}&\frac{78+2\sqrt{761}}{\sqrt{1522-42\sqrt{761}}\sqrt{29+\sqrt{761}}}\\\frac{-2-2\sqrt{761}}{\sqrt{1522+42\sqrt{761}}\sqrt{29-\sqrt{761}}}&\frac{-2+2\sqrt{761}}{\sqrt{1522-42\sqrt{761}}\sqrt{29+\sqrt{761}}}}   
\end{align}
Now, to calculate \vec{x}
\begin{align}
  \vec{M}\vec{x}=\vec{b}\\
  \implies \vec{U}\vec{S}\vec{V}^T\vec{x}=\vec{b}\\
  \implies \vec{S}\vec{V}^T\vec{x}=\vec{U}^T\vec{b}\\
  \implies \vec{x}=\vec{V}(\vec{S_+}(\vec{U}^T\vec{b}))
\end{align}
Calculating $\vec{U}^T\vec{b}$, we have
\begin{align}
  \myvec{\frac{45+\sqrt{761}}{\sqrt{1522+42\sqrt{761}}}\\\frac{45-\sqrt{761}}{\sqrt{1522-42\sqrt{761}}}} \\
 \vec{S_+}(\vec{U}^T\vec{b})=\myvec{\frac{45+\sqrt{761}}{\sqrt{1522+42\sqrt{761}}\sqrt{29-\sqrt{761}}}\\\frac{45-\sqrt{761}}{\sqrt{1522-42\sqrt{761}}\sqrt{29+\sqrt{761}}}}
 \end{align}
 $\vec{V}(\vec{S_+}(\vec{U}^T\vec{b}))$
 \begin{align}
 =\myvec{\frac{78-2\sqrt{761}}{\sqrt{1522+42\sqrt{761}}\sqrt{29-\sqrt{761}}}&\frac{78+2\sqrt{761}}{\sqrt{1522-42\sqrt{761}}\sqrt{29+\sqrt{761}}}\\\frac{-2-2\sqrt{761}}{\sqrt{1522+42\sqrt{761}}\sqrt{29-\sqrt{761}}}&\frac{-2+2\sqrt{761}}{\sqrt{1522-42\sqrt{761}}\sqrt{29+\sqrt{761}}}} \\
 \myvec{\frac{45+\sqrt{761}}{\sqrt{1522+42\sqrt{761}}\sqrt{29-\sqrt{761}}}\\\frac{45-\sqrt{761}}{\sqrt{1522-42\sqrt{761}}\sqrt{29+\sqrt{761}}}}
\end{align}
Solving, 
\begin{align}
  \vec{x}=\myvec{\frac{8371}{15220}\\\frac{-15981}{15220}}=\myvec{\frac{11}{20}\\\frac{-21}{20}}  
\end{align}
Verifying the solution,
\begin{align}
    \vec{M}\vec{x}=\vec{b}\\
    \implies \vec{M}^T\vec{M}\vec{x}=\vec{M}^T\vec{b}\label{5}\\
    \vec{M}^T\vec{b}=\myvec{3&2&6\\1&2&2}\myvec{5\\-1\\-1}\\
    =\myvec{7\\1}\\
    \vec{M}^T\vec{M}=\myvec{3&2&6\\1&2&2}\myvec{3&1\\2&2\\6&2}\\
    =\myvec{49&19\\19&9}\\
\text{From, $\eqref{5}$ }
\myvec{49&19\\19&9}\vec{x}= \myvec{7\\1} 
\end{align}
Solving for $\vec{x}$
\begin{align}
  \myvec{49&19&7\\19&9&1}\xleftrightarrow[]{R_2\leftarrow R_2- R_1\times \frac{19}{49}}\myvec{49&19&7\\0&\frac{80}{49}&\frac{-84}{49}}\\
  \xleftrightarrow[]{R_1\leftarrow R_1\times \frac{1}{49}}\myvec{1&\frac{19}{49}&\frac{7}{49}\\0&\frac{80}{49}&\frac{-84}{49}}
  \\\xleftrightarrow[]{R_1\leftarrow R_1- R_2\times \frac{19}{80}}\myvec{1&0&\frac{11}{20}\\0&\frac{80}{49}&\frac{-84}{49}}\\
  \xleftrightarrow[]{R_2\leftarrow R_2\times \frac{49}{80}}\myvec{1&0&\frac{11}{20}\\0&1&\frac{-21}{20}}
  \\ \implies \vec{x}=\myvec{\frac{11}{20}\\\frac{-21}{20}}
\end{align}
\end{document}

